%\documentclass[10pt,conference]{IEEEtran}
%\documentclass[10pt, final, journal, letterpaper, twocolumn]{IEEEtran}
\documentclass[12pt, draftclsnofoot, journal, letterpaper, onecolumn]{IEEEtran}

%\documentclass[a4paper,11pt]{article}

%\documentclass[conference]{IEEEtran}
\makeatletter
\def\ps@headings{%
\def\@oddhead{\mbox{}\scriptsize\rightmark \hfil \thepage}%
\def\@evenhead{\scriptsize\thepage \hfil \leftmark\mbox{}}%
\def\@oddfoot{}%
\def\@evenfoot{}}
\makeatother \pagestyle{headings}

\IEEEoverridecommandlockouts
\usepackage{bbm}
\usepackage{amsfonts}
\usepackage[dvips]{graphicx}
\usepackage{times}
\usepackage{cite}
\usepackage{amsmath}
\usepackage{array}
%\usepackage{chgbar}
\usepackage{amssymb}
\newcommand{\mb}{\mathbf}
\newcommand{\bs}{\boldsymbol}
\newcommand{\sss}{\scriptscriptstyle}
\newcounter{mytempeqncnt}
\usepackage{stfloats}
\usepackage{slashbox}
\usepackage{graphicx}
\usepackage{footnote}
%\usepackage{amsthm}
\usepackage{booktabs}
\usepackage{array}
%\usepackage{algorithmic}
\usepackage{algorithm}
\usepackage{subeqnarray}
\usepackage{cases}
\usepackage{threeparttable}
\usepackage{color}
\usepackage{hyperref}
\usepackage{epstopdf}
\usepackage{algpseudocode}
\usepackage{bm}
%\usepackage{subcaption}
\usepackage{multirow}
% to add parentheses around subfig references
\usepackage[labelformat=simple]{subcaption}
%\usepackage[labelfont=small]{caption}
\usepackage{adjustbox}
%http://tex.stackexchange.com/questions/163246/resize-a-tabular-object-to-textwidth
\renewcommand\thesubfigure{(\alph{subfigure})}
\newtheorem{theorem}{Theorem}
\newtheorem{remark}{Remark}
\newtheorem{proposition}{Proposition}
\newtheorem{lemma}{Lemma}
\newtheorem{definition}{Definition}
\newtheorem{problem}{Problem}
\newtheorem{corollary}{Corollary}

% line break in a table
\newcommand{\tabincell}[2]{\begin{tabular}{@{}#1@{}}#2\end{tabular}}
\begin{document}

\title{Review of Reconfigurable Intelligent Surface}


\author{\authorblockN{Tianzheng Miao}\\
\authorblockA{SSE, Chinese University of HongKong, Shenzhen
}
}

\maketitle

\vspace{-1.5cm}
\section{Introduction}
The demand for wireless communication services has been shifting from connection-oriented services such as, traditional voice telephony and messaging to content-oriented services such as digital media, social networking and smartphone applications.
This phenomenon propels the development of content-centric wireless networks\cite{liuhui}.
Recently, to support the dramatic growth of the wireless data traffic, caching at base stations (BSs) has been proposed as a promising approach for massive content delivery\cite{femto}. Moreover, enabling multicast service at BSs is an efficient way to deliver contents to multiple requesters simultaneously\cite{embms}.
\section{System Model}
We consider a IRS-based target detection system as illustrated in Fig. 1, where a base station(BS) can detect a moving car based on echo signal reflected by the IRS that consists of $L$ passive reflecting elements fixed on the car. Over $N$ observations, BS consecutively sends single directional probing signals and IRS forms different single directional beams which can cover the whole space. The passive reflecting states are drawn from a pre-designed codebook \textbf{$\Phi$}. Let \textbf{$\Phi$}
% \centering
% \begin{figure*}
% \includegraphics[width=0.75\textwidth]{systemmodel.png}
% \caption{PSD diagram}
% \label{fig:1}       % Give a unique label
% \end{figure*}

\bibliographystyle{IEEEtran}
\bibliography{review}

\end{document}
